\chapter*{Appendix}
\markboth{Appendix}{Appendix}
\markright{Appendix}

\section*{Wildlife Strike Risk Index}
In this thesis work a further model proposed by BirdControl Italy S.r.l. has been implemented.
This model, the Wildlife Strike Risk Index ($WRI$), has been proposed to replace the current $BRI_2$ with the purpose of:

\begin{itemize}
    \item Simplicity and clarity of calculation
    \item Objectivity and standardization
    \item Independence from daily monitoring of fauna
    \item Definition of an objective attention threshold
\end{itemize}
The $WRI$ has been defined on the basis of the following 5 components:

\begin{itemize}
    \item $C1$. Ratio between the Wildlife Strike ($WS$) events and number of movements
    \item $C2$. Species involved in wildlife strikes 
    \item $C3$. No. Individuals involved for $WS$
    \item $C4$. Effect on flight and affected part of the aircraft
    \item $C5$. Efficiency monitoring strategy
\end{itemize}
Calculated the 5 components, you can estimate the $WRI$ with the following formula in equation \ref{WRI_Formula} :

\begin{equation}\label{WRI_Formula}
WRI= \frac{\sum_{i=1}^5 C_{i}}{100}
\end{equation}

\subsubsection{C1}

This component is a measure of risk frequency, and is calculated by comparing the number of wildlife strike events with the number of movements ($mov$) within the airport during the month.
An airport that records 20 or more $WS$ for every 1000 movements (2\% or more) is assigned a maximum value of 25 (this ensures that the $WRI$ takes a maximum value of 1), therefore:

\begin{equation}\label{to_c1_1}
20 : 1000 = 25 : X \\
\end{equation}
\begin{equation}\label{to_c1_2}
X = (1000 \cdot 25)/20 = 1250
\end{equation}
Finally, given the equations \ref{to_c1_1} and \ref{to_c1_2}, $C1$ is computed with the following formula in equation \ref{C1}:

\begin{equation}\label{C1}
C1 = \frac{\frac{\mathrm{WS}}{\mathrm{month}*1250}}{\mathrm{mov}/\mathrm{month}}
\end{equation}

\subsubsection{C2}
This component is a measure of the severity of the risk, and is calculated on the basis of a heuristic matrix \cite{allan2006heuristic}, associating a value between 1 and 25 to the species involved in each event.

\subsubsection{C3}
This component is a measure of the severity of the risk, and is calculated by associating a value (from 1 to 5) depending on the number of individuals involved in a wildlife strike event, shown in the Table \ref{tab-c3}.

\begin{table}
	\centering
	\scalebox{1}{
	\begin{tabular}{@{}ccc@{}}
		\toprule
		Number of Animals Involved & Severity of the Risk \\	\midrule
		1 & 1 \\
		From 2 to 10 & 2 \\
		From 11 to 100 & 3\\
		$>$ 100 &  5 \\	\bottomrule
	\end{tabular}}
	\caption{$C3$: measure of the severity of the risk. The table lists in ascending order the hazard values attributed to the number of animals involved in wildlife strike events.}
	\label{tab-c3}
\end{table}

\subsubsection{C4}
This component is a measure of the severity of the risk whose values range from 4.3 to 30 depending on the severity of the $EOF$.
In particular, in case the aircraft is hit and/or damaged the $C4$ value is doubled and in case of a catastrophic event, the $WRI$ value related to that event shall be set to the default value1.
The Table \ref{tab-EOF_wri} show the $EOF$ values:

\subsubsection{C5}

This component is a measure of risk frequency, and is calculated based on the number of unscheduled wildlife strike inspections ($UI$) at the individual airport.
An airport that records 60 or more $UI$ per month has a maximum value of 15 (this ensures that the $WRI$ is at a maximum value of 1), therefore:

\begin{equation}\label{to_c5_1}
60 : 15 = \mathrm{UI} : X
\end{equation}
\begin{equation}\label{to_c5_2}
 X = \frac{15 \cdot \mathrm{UI}}{60}
\end{equation}
Finally, given the equations \ref{to_c5_1} and \ref{to_c5_2}, $C5$ is computed with the following formula in equation \ref{C5}:

\begin{equation}\label{C5}
C5 = \frac{\mathrm{UI} \cdot 15}{60}
\end{equation}

\begin{table}
	\centering
	\scalebox{1}{
	\begin{tabular}{@{}ccc@{}}
		\toprule
		EOF Value & Description \\	\midrule
		4.3 & None \\
		5 & Aborted take-off, delay \\
		6 & Precautionary landing \\
		7.5 &  Engine(s) shutdown \\
		10 & Forced landing \\	
		15 & Visual obstruction \\	
		- & Catastrofic effect \\	\bottomrule
	\end{tabular}}
	\caption{Categories of the Effect On Flight ($EOF$) provoked by wildlife strike events. The table lists in ascending order the values attributed to damage caused by wildlife events.}
	\label{tab-EOF_wri}
\end{table}

By summing up the scores for the five variables presented, the $WRI$ is obtained (eq. \ref{WRI_Formula}). The value is divided by 100, resulting in a value between 0 and 1.
The weighted arithmetic mean of the events recorded in a month gives the monthly $WRI$.
Similarly, the weighted arithmetic mean of the events recorded in a year gives the annual $WRI$.

This model was developed and tested on a subset of airports.
Two main problems were detected:
\begin{itemize}
    \item the first problem is the poor correlation of the $WRI$ with historical birdstrike events,
    \item the second problem is the low risk values that the $WRI$ provides over many years of computing, failing to report the real periods of danger.
\end{itemize}
From a first analysis the $WRI$ is not an accurate risk index and a reliable risk predictor, having in common the poor correlation with birdstrike events as the $BRI_2$ showed. 


