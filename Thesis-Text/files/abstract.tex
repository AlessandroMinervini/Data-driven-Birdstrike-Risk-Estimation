\chapter*{Abstract}
Wildlife strike events (or more commonly, birdstrikes) are a dangerous and growing phenomenon due to the increase in flights over the years. Unfortunately, airports are attractive places for birds and other animals which can cause serious wildlife strike accidents.

This phenomenon can be seriously dangerous for the safety of passengers and very expensive for civil aviation, not only in Italy, to sustain the cost of prevention, repair and delays caused.
To keep airports safe it becomes very important to prevent these events.

At the moment in all Italian airports the Birdstrike Risk Index version 2.0 ($BRI_2$) is used to monitor risk of wildlife strike events. It provides a monthly value that identifies the risk to which each airport is subject. 

We conducted an in-depth analysis of $BRI_2$ to estimate the quality of this risk index, its correlation with historically occurring wildlife strike events, in order to understand if it is a good tool for measuring actual wildlife risk.
Some limitations of the $BRI_2$ have been found. A risk-index for this phenomenon must show a relevant correlation with wildlife strike events. 

In this thesis we develop a data-driven model for wildlife strike risk estimation. Our approach is based on a Recurrent Neural Network (RNN) trained to predict a wildlife strike risk estimate that is as correlate as possible with historically occurring wildlife strike events in many Italian Airports, on the basis of a fifteen-day window of data.
Our evaluation, using data from a subset of eight different airports, show encouraging results achieved by our approach demonstrating the superior predictive quality of our model with respect to $BRI_2$.