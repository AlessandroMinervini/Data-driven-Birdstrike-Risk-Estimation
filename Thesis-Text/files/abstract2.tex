\chapter*{Sommario}

Gli eventi wildlife strike (o più comunemente birdstrike) sono un pericoloso fenomeno in crescita a causa dell'aumento dei voli aerei nel corso degli anni. Gli aeroporti sono purtroppo luoghi attrattivi per uccelli ed altri animali i quali possono provocare incidenti di wildlife strike anche di seria entità.
Questo fenomeno può essere seriamente pericoloso per la sicurezza dei passeggeri e molto costoso per l'aviazione civile, non solo italiana, sostenere i costi di prevenzione, di riparazione e per i ritardi provocati.
Per mantenere gli aeroporti sicuri diventa molto importante prevenire questi eventi.

Al momento in tutti gli aeroporti italiani per monitorare il rischio di wildlife strike è utilizzato il Birdstrike Risk Index version 2.0 ($BRI_2$). Esso fornisce un valore mensile che identifica il rischio a cui è soggetto ogni aeroporto. 
Un'analisi profonda è stata condotta sul $BRI_2$ stimando la qualità di questo indice di rischio, e la correlazione con gli eventi di wildlife strike storicamente accaduti, per capire se è un buon strumento capace di riconoscere i periodi di maggiore pericolosità di questo fenomeno.
Alcune limitazioni del $BRI_2$ sono state rilevate. Un indice di rischio per questo fenomeno deve manifestare una rilevante correlazione con gli eventi di wildlife strike.

In questa tesi è stato sviluppato un modello data-driven per la stima del rischio degli eventi di wildlife strike. Il nostro approccio è basato su una Recurrent Neural Network (RNN) allenata a predirre una stima del rischio di wildlife strike che sia il più correlato possibile con gli eventi di wildlife storicamente accaduti sulla base di una finestra di osservazioni di 15 giorni.
La nostra valutazione, usando dati provenienti da otto diversi aeroporti, mostrano risultati incoraggianti raggiunti dal nostro approccio, dimostrando la superiore qualità predittiva del nostro modello rispetto al $BRI_2$.